\begin{abstract}
\noindent
This thesis analyzes one hour based energy disaggregation using Sparse Coding by exploiting temporal differences. Energy disaggregation is the task of taking a whole-home energy signal and separating it into its component appliances. Studies have shown that having device-level energy information can cause users to conserve significant amounts of energy, but current electricity meters only report whole-home data. Thus, developing algorithmic methods for disaggregation presents a key technical challenge in the effort to maximize energy conservation. In Energy Disaggregation or sometimes called Non-Intrusive Load Monitoring (NILM) most approaches are based on high frequent monitored appliances, while households only measure their consumption via smart-meters, which only account for one-hour measurements. This thesis aims at implementing key algorithms from J. Zico Kotler, Siddarth Batra and Andrew Ng paper "Energy Disaggregation via Discriminative Sparse Coding" and try to replicate the results by exploiting temporal differences that occur when dealing with time series data. The implementation was successful, but the results were inconclusive when dealing with large datasets, as the algorithm was too computationally heavy for the resources available. The work was performed at the Swedish company Greenely, who develops visualizations based on gamification for energy bills via a mobile application.


%\\
%It is shown that blah blah
%It is shown that all such games are numbers and that the dominating game options are to remove elements not lower than any other element of the same color. 
%\\
%In particular, the thesis show that ...
%In particular, the thesis concerns games played on posets that are chess-colored Young diagrams. It is shown that it is easy to compute the value for any such game with $\le3$ rows by proving a proposed formula for computing the value.
\end{abstract}
%~\\\\
\newpage
\rule{\linewidth}{0.2mm}
{\begin{spacing}{1.5} \centering{\LARGE\bfseries  Utforskande av tempor\"ara skillnader f\"or Energi Disaggregering med Sparse Coding} \end{spacing}}
\rule{\linewidth}{0.2mm}\\[1.5cm]

\renewcommand{\abstractname}{Sammanfattning}
\begin{abstract}
\noindent
I den h\"ar uppsatsen analyseras Energi Disaggregering med hjälp av Sparse Coding genom att utforska temporala skillnader på en timbaserade data. Studer har visat att presentera information p\r{a} apparat nivå kan det g\"ora att anv\"andare utnyttjar mindre energi i hemmen d\"ar man idag bara presenterar hela hush\r{a}llets anv\"andning. Denna uppsats har som mål att implementera och utveckla de algoritmer J. Zico Kotler, Siddarth Batra and Andrew Ng presenterar i deras artikel "Energy Disaggregation via Discriminative Sparse Coding", genom att anv\"anda sig utav de tempor\"ara skillnader som uppstår inom tidsseriedata. I uppsatsen uppnåde man ett s\"amre resultat, d\"ar datakraften var inte tillr\"acklig f\"or att utnyttja datan på b\"asta s\"att. Arbetet var utf\"ort på Greenely, ett f\"oretag som utvecklar visualiseringar utav elr\"akningen via en mobilapplikation.
%I synnerhet fokuserar denna uppsats p\r{a} spel som spelas p\r{a} pom\"angder som \"ar schackf\"argade Young-diagram. Det visas att det \"ar l\"att att ber\"akna v\"ardet p\r{a} alla s\r{a}dana spel med $\le3$ rader genom att bevisa en f\"oreslagen formel f\"or att r\"akna ut v\"ardet.
\end{abstract}
\newpage
%\newpage
\renewcommand{\abstractname}{Acknowledgements}
\begin{abstract}
\noindent
I would like to express my gratitude to my supervisor at KTH Royal Institute of Technology, Timo Koski, for his valuable scientific support and continuous interest and dedication throughout the process of this thesis. I would also like to thank Pawel Herman, for his interest in my thesis. I take this opportunity to express my sincerest gratitude to all of the team members at Greenely for providing me with the facilities, equipment, support and their knowledge within the field.
\end{abstract}