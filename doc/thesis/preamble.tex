\usepackage[utf8]{inputenc}
\usepackage[english]{babel}
\usepackage{amsmath,amsfonts,amssymb,amsthm,amscd}
\usepackage{mathtools}
\usepackage{color}
\usepackage{parskip}
\usepackage{fancyhdr}
\usepackage{nicefrac}
\usepackage{cite}
\usepackage{graphicx}
\usepackage{subcaption}
\usepackage[table]{xcolor}
\usepackage{tikz}
\usetikzlibrary{decorations.markings}
\usetikzlibrary{calc}
\usetikzlibrary{shapes.misc}
\usetikzlibrary{hobby} 
\tikzstyle{hackennode}=[draw,circle,fill=white,inner sep=0,minimum size=4pt]
\tikzstyle{hackenline}=[line width=3pt]
\usepackage{scalefnt}
\usepackage[all]{xy}
\usepackage{pgfplots}
\usepackage{listings}
\usepackage{setspace}
\usepackage{siunitx}
\usepackage{float}
\usepackage{textcomp}
\usepackage{caption}
\usepackage[nottoc,notlot,notlof]{tocbibind}
\usepackage{mdframed}
\usepackage{hyperref}
\usepackage{dsfont}

% norm
\newcommand{\norm}[1]{\left\lVert#1\right\rVert}

% how to footnotes across 2 pages
\interfootnotelinepenalty=10000

\theoremstyle{plain}
\newtheorem{thm1}{Theorem}
\newtheorem{lem1}[thm1]{Lemma}
\newtheorem{cor}[thm1]{Corollary}
\newtheorem{prop}[thm1]{Proposition}
\newtheorem{ctex}[thm1]{Counterexample}

\theoremstyle{definition}
\newtheorem{ex1}[thm1]{Example}
\newtheorem{remrk}[thm1]{Remark}
\newtheorem{defn1}[thm1]{Definition}

\usepackage[many]{tcolorbox}

\newtcolorbox{ex}[2][]{%
  enhanced,frame hidden,interior hidden,
  arc=10pt,outer arc=10pt,borderline={3pt}{2pt}{dashed,thick},borderline={3pt}{-2pt}{dashed,thick},
  before upper={#2\begin{ex1}},after upper={\end{ex1}}
}
\newtcolorbox{proof2}[2][]{%
  enhanced,breakable,frame hidden,interior hidden,
  arc=0pt,outer arc=0pt,borderline={0.5pt}{-0pt}{dashed},
  before upper={\begin{proof}[\textbf{#2}]},after upper={\end{proof}}
}

\newenvironment{defn}
  {\begin{mdframed}\begin{defn1}}
  {\end{defn1}\end{mdframed}}
\newenvironment{thm}
  {\begin{mdframed}\begin{thm1}}
  {\end{thm1}\end{mdframed}}
\newenvironment{lem}
  {\begin{mdframed}[nobreak=true]\begin{lem1}}
  {\end{lem1}\end{mdframed}}

\usepackage[a4paper, total={5.6in, 8.2in}]{geometry}
\usepackage{wrapfig}
\usepackage[labelfont=bf]{caption}
\usepackage{rotating}

%\allowdisplaybreaksfind

\usepackage{lipsum}
\usepackage{hhline}

% - Algorithms --
\usepackage{algpseudocode}% http://ctan.org/pkg/algorithmicx
\usepackage[titlenumbered,ruled]{algorithm2e}
\usepackage{algcompatible}
%

% - Python source code --
\usepackage{minted}
%