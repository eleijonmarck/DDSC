\begin{abstract}
\noindent
This thesis analyzes a class of element-removal partizan games played on colored posets. In these games a player moves by removing an element of its color together with all greater elements in the poset. A player loses if it has no elements left to remove. 
\\
It is shown that all such games are numbers and that the dominating game options are to remove elements not lower than any other element of the same color. 
\\
In particular, the thesis concerns games played on posets that are chess-colored Young diagrams. It is shown that it is easy to compute the value for any such game with $\le3$ rows by proving a proposed formula for computing the value.
%\lipsum[1-2]
\end{abstract}
~\\\\
%\newpage
\renewcommand{\abstractname}{Sammanfattning}
\begin{abstract}
\noindent
I den h\"ar uppsatsen analyseras en klass av partiska spel som spelas på f\"argade pom\"angder. Spelen spelas i omg\r{a}ngar mellan tv\r{a} spelare d\"ar spelaren under sin tur v\"aljer ut ett element i pom\"angden som \"ar i spelarens f\"arg och avl\"agsnar det elementet och alla st\"orre element i pom\"angden. En spelare f\"orlorar om den inte l\"angre har n\r{a}got element att avl\"agsna.
\\
I uppsatsen visas det att alla s\r{a}dana spel \"ar tal och att de dominerande spelalternativen \"ar att avl\"agsna element som inte \"ar mindre \"an n\r{a}got annat element av samma f\"arg.
\\
I synnerhet fokuserar denna uppsats p\r{a} spel som spelas p\r{a} pom\"angder som \"ar schackf\"argade Young-diagram. Det visas att det \"ar l\"att att ber\"akna v\"ardet p\r{a} alla s\r{a}dana spel med $\le3$ rader genom att bevisa en f\"oreslagen formel f\"or att r\"akna ut v\"ardet.
\end{abstract}
\newpage
%\newpage
\renewcommand{\abstractname}{Acknowledgements}
\begin{abstract}
\noindent
I would like express my gratitude to my supervisor at Royal Institute of Technology, Jonas Sj\"ostrand, for his valuable support and continuous interest and dedication throughout the process of this thesis and for keeping me inspired and devoted even after the worst of setbacks.
\end{abstract}